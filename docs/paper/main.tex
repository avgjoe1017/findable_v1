% =============================================================================
% Measuring AI Sourceability: The Findable Score Framework
% arXiv preprint — February 2026
% =============================================================================

\documentclass[11pt,a4paper]{article}

% --- Standard packages -------------------------------------------------------
\usepackage[utf8]{inputenc}
\usepackage[T1]{fontenc}
\usepackage{amsmath,amssymb}
\usepackage{graphicx}
\usepackage[colorlinks=true,linkcolor=blue,citecolor=blue,urlcolor=blue]{hyperref}
\usepackage{booktabs}
\usepackage{natbib}
\usepackage{xcolor}
\usepackage[margin=1in]{geometry}
\usepackage{enumitem}
\usepackage{multirow}
\usepackage{array}
\usepackage{float}
\usepackage{algorithm}
\usepackage{algpseudocode}

% --- Custom commands ----------------------------------------------------------
\newcommand{\findable}{\textsc{Findable Score}}
\newcommand{\pillar}[1]{\textsc{#1}}
\newcommand{\obs}[1]{\texttt{obs\_#1}}

% ==============================================================================
\title{Measuring AI Sourceability: A Multi-Pillar Framework for Predicting\\
Whether AI Answer Engines Can Retrieve and Cite Web Content}

\author{
  Joe Barrett\thanks{Corresponding author. Email: joe@findablescore.com}\\
  Findable Score\\
}

\date{February 2026 \\ \medskip \small Working Paper}

\begin{document}
\maketitle

% ==============================================================================
% ABSTRACT
% ==============================================================================
\begin{abstract}
As AI answer engines (ChatGPT, Perplexity, Google AI Overviews, Gemini)
increasingly replace traditional search as the primary discovery mechanism for
web content, a critical measurement gap has emerged: no standardized methodology
exists for predicting whether a given website will be retrieved and cited by
these systems. We present the \findable{}, a 0--100 composite metric built on a
7-pillar scoring framework that measures \emph{AI sourceability} --- the degree
to which a website's content is accessible, parseable, retrievable, and citable
by AI answer engines. Our methodology combines deterministic simulation (crawl,
extract, chunk, embed, retrieve, grade) with observed reality snapshots from
live AI model outputs, calibrated against 1,275+ ground-truth samples across
35+ domains. We introduce three novel contributions: (1)~the empirical
distinction between \emph{mentions} (brand name appears in AI output; 99.8\%
positive rate, statistically useless) and \emph{citations} (URL explicitly
cited; 68\% positive rate, predictively meaningful) as calibration targets;
(2)~a Citation Context layer that classifies websites into 9 content types with
empirically-derived citation baselines, explaining why technically
well-optimized sites can receive zero citations; and (3)~a Source Primacy
classification (\textsc{Primary}/\textsc{Authoritative}/\textsc{Derivative})
that resolves the Citation Paradox --- the counterintuitive finding that
low-citation sites often have \emph{higher} technical scores than high-citation
sites. We report 80.8\% training accuracy and 50.7\% holdout accuracy using
domain-stratified validation, and discuss why the 30-point gap reveals
fundamental limits of website-side features in predicting AI citation behavior.
% TODO: Update with final calibration numbers after recalibration
\end{abstract}

\bigskip
\noindent\textbf{Keywords:} AI sourceability, generative engine optimization,
retrieval-augmented generation, web content citation, findability scoring

% ==============================================================================
% 1. INTRODUCTION
% ==============================================================================
\section{Introduction}\label{sec:introduction}

\subsection{The Shift from Search to Answer Engines}\label{sec:shift}

The information retrieval landscape is undergoing a structural transformation.
AI answer engines --- systems that generate natural language responses grounded
in retrieved web content --- are rapidly displacing traditional keyword-based
search. As of early 2026, ChatGPT processes approximately 2.5 billion prompts
daily across 700 million weekly active users. Google AI Overviews appear in
approximately 60\% of US search queries, up from roughly 30\% in mid-2024.
Perplexity serves 33 million monthly active users. Collectively, these systems
process over 4 billion prompts per day.

This shift creates a new class of problem for website operators. Traditional
search engine optimization (SEO) focuses on ranking signals --- backlinks,
keyword density, page speed, mobile-friendliness --- that determine position in
a list of ten blue links. AI answer engines operate differently: they retrieve
content, synthesize it into a coherent response, and optionally \emph{cite} the
source. The website either appears as a cited source or it does not. There is no
``rank~7'' --- there is only cited or not cited, mentioned or not mentioned.

Despite the magnitude of this shift, no standardized methodology exists for
measuring whether a given website is likely to be retrieved and cited by AI
answer engines. Existing tools either track surface-level metrics (is the brand
name mentioned?) or operate as opaque scoring systems without disclosed
methodology. This paper presents a transparent, reproducible framework for
measuring AI sourceability.

\subsection{Definitions}\label{sec:definitions}

\begin{description}[style=unboxed,leftmargin=0pt]
  \item[AI Sourceability.] The degree to which a website's content can be
    accessed, parsed, retrieved, and cited by AI answer engines.
  \item[\findable{}.] A composite 0--100 metric quantifying AI sourceability
    across 7 structural pillars.
  \item[Mention.] The brand name or product name appears in an AI-generated
    response. (Example: ``Datadog is a monitoring platform.'')
  \item[Citation.] The AI response explicitly references the website's URL as a
    source. (Example: ``According to Datadog's documentation
    [datadog.com/docs/\ldots]'')
  \item[Source Primacy.] Whether a website is \emph{the} canonical source for
    its content (\textsc{Primary}), a recognized authority
    (\textsc{Authoritative}), or one of many equivalent sources
    (\textsc{Derivative}).
\end{description}

\subsection{Contributions}\label{sec:contributions}

We make five contributions:
\begin{enumerate}
  \item A transparent 7-pillar scoring framework for AI sourceability with
    disclosed algorithms, weights, and thresholds.
  \item The empirical demonstration that mention-tracking (99.8\% positive rate)
    is statistically uninformative, while citation-tracking (68\% positive rate)
    provides meaningful signal.
  \item A two-layer scoring model separating \emph{sourceability} (what the
    website controls) from \emph{citation likelihood} (what depends on content
    type and competitive context).
  \item A Source Primacy classification system that resolves the Citation
    Paradox.
  \item A domain-stratified calibration methodology using grid search
    optimization with bias-adjusted metrics.
\end{enumerate}

% ==============================================================================
% 2. RELATED WORK
% ==============================================================================
\section{Related Work}\label{sec:related}

\subsection{Traditional SEO Scoring}\label{sec:seo-scoring}

Moz's Domain Authority (DA) and Ahrefs' Domain Rating (DR) are the most
widely-adopted website quality scores. Both predict Google search ranking based
primarily on backlink profiles. Neither attempts to predict AI answer engine
behavior, and both operate as proprietary black-box systems without disclosed
methodology.

\subsection{Generative Engine Optimization}\label{sec:geo}

\citet{aggarwal_2024_geo} introduced the term ``Generative Engine Optimization''
at KDD~2024, proposing a framework for optimizing content visibility in
generative search engines. Their work focuses on content-level optimization
strategies (adding statistics, citing sources, using authoritative language)
rather than systematic measurement of sourceability across structural
dimensions.

\subsection{AI Visibility Tracking Tools}\label{sec:ai-tools}

A growing ecosystem of commercial tools tracks AI visibility: Profound
(prompt-level tracking with 400M+ real prompts), Peec AI (auto-generated prompt
monitoring), Otterly.ai (multi-platform mention tracking), and others. These
tools primarily track \emph{outcomes} (was the brand mentioned or cited in a
specific prompt?) rather than \emph{causes} (what structural properties of the
website determine citation likelihood?). None disclose their scoring
methodology.

\subsection{Retrieval-Augmented Generation}\label{sec:rag}

The RAG paradigm \citep{lewis_2020_rag} provides the architectural foundation
for understanding how AI answer engines interact with web content. The retrieval
stage --- where content is selected from an index based on query relevance ---
is the critical bottleneck that our simulation targets. Our framework simulates
this retrieval stage using the same class of techniques (embedding-based vector
search, BM25 lexical matching, rank fusion) employed by production RAG systems.

% ==============================================================================
% 3. METHODOLOGY
% ==============================================================================
\section{Methodology}\label{sec:methodology}

\subsection{System Architecture}\label{sec:architecture}

The \findable{} is computed through a 15-step audit pipeline, divided into two
pathways:

\begin{itemize}
  \item \textbf{Simulated Findability} (Steps 1--13): A deterministic,
    reproducible assessment requiring no external API calls to LLM providers.
    Encompasses technical checks, crawling, extraction, chunking, embedding,
    question generation, retrieval simulation, scoring, pillar analysis,
    composite scoring, fix generation, and report assembly.
  \item \textbf{Observed Findability} (Steps 14--15): Reality snapshots from
    actual AI model outputs used for validation and calibration.
\end{itemize}

% TODO: Consider adding a pipeline diagram as a figure here
% \begin{figure}[ht]
%   \centering
%   \includegraphics[width=\textwidth]{figures/figure_pipeline.pdf}
%   \caption{The 15-step audit pipeline.}
%   \label{fig:pipeline}
% \end{figure}

\subsection{Content Extraction and Chunking}\label{sec:chunking}

Content is extracted using trafilatura \citep{barbaresi_2021_trafilatura} with
BeautifulSoup as fallback for pages where trafilatura's heuristics fail.
Extracted content is split into semantically-coherent chunks with the parameters
shown in Table~\ref{tab:chunking}.

\begin{table}[ht]
  \centering
  \caption{Chunking parameters.}
  \label{tab:chunking}
  \begin{tabular}{@{}lrl@{}}
    \toprule
    \textbf{Parameter} & \textbf{Value} & \textbf{Rationale} \\
    \midrule
    Max chunk size  & 512 tokens  & Optimal for bge-small-en-v1.5 context window \\
    Min chunk size  & 100 tokens  & Prevents degenerate fragments \\
    Overlap         & 50 tokens   & Preserves cross-boundary context \\
    \bottomrule
  \end{tabular}
\end{table}

Unlike naive fixed-size chunking, our splitter detects and preserves content
type boundaries. Tables are detected by pipe-delimited formatting ($>$50\% of
lines contain pipes), lists by bullet/number prefix patterns, code blocks by
fenced code markers, and blockquotes by~\texttt{>}~prefix patterns. Each chunk
stores its nearest heading context and its position ratio within the source
document, enabling context-aware retrieval. Duplicate chunks are prevented via
MD5 hashes of normalized content.

\subsection{Question Generation}\label{sec:questions}

Each audit generates approximately 20 test questions across two sources:

\paragraph{Universal questions (15).}
Five categories --- Identity~(3), Offerings~(4), Contact~(2), Trust~(3),
Differentiation~(3) --- with per-question difficulty weights ranging from 1.0
to~1.5. These represent the core information-seeking queries that AI answer
engines are most likely to encounter about any business entity.

\paragraph{Site-derived questions (up to 5).}
Generated from the website's own signals: Schema.org structured data (Product,
LocalBusiness, SoftwareApplication schemas trigger type-specific questions) and
heading analysis (headings matching pricing, features, or capability patterns
generate targeted questions). Maximum 3 schema-derived plus 2 heading-derived
questions prevent over-indexing on site-specific content.

\subsection{Hybrid Retrieval Simulation}\label{sec:retrieval}

For each test question, we simulate the retrieval stage of a RAG pipeline using
hybrid search combining BM25 lexical matching with dense vector retrieval.

\paragraph{BM25 scoring.}
The BM25 score for a document~$d$ given query~$q$ is:
\begin{equation}\label{eq:bm25}
  \text{BM25}(d, q)
  = \sum_{t \in q}
    \text{IDF}(t) \cdot
    \frac{
      \text{tf}(t, d) \cdot (k_1 + 1)
    }{
      \text{tf}(t, d) + k_1 \cdot
      \bigl(1 - b + b \cdot |d| / \text{avgdl}\bigr)
    }
\end{equation}
where $k_1 = 1.5$ (term frequency saturation) and $b = 0.75$ (document length
normalization), and IDF is computed as:
\begin{equation}\label{eq:idf}
  \text{IDF}(t) = \log\!\left(\frac{N - \text{df}(t) + 0.5}{\text{df}(t) + 0.5} + 1.0\right)
\end{equation}

\paragraph{Vector search.}
Cosine similarity between query embedding and chunk embeddings, using
bge-small-en-v1.5 \citep{xiao_2023_bge} (384-dimensional, normalized).

\paragraph{Reciprocal Rank Fusion (RRF).}
Results from both methods are fused using RRF
\citep{cormack_2009_rrf}:
\begin{equation}\label{eq:rrf}
  \text{RRF}(d) = \sum_{s \in \{\text{bm25}, \text{vector}\}}
    \frac{w_s}{k + \text{rank}_s(d)}
\end{equation}
with $k = 60$, $w_{\text{bm25}} = 0.5$, and $w_{\text{vector}} = 0.5$. Top~20
results from each method are fused, with a diversity constraint (maximum 2
chunks per source page) applied before selecting the final top~10 results.

\subsection{The Seven Pillars}\label{sec:pillars}

The \findable{} aggregates seven pillar scores, each measuring a distinct
dimension of AI sourceability. Table~\ref{tab:pillar-weights} summarizes the
default weights.

\begin{table}[ht]
  \centering
  \caption{Default pillar weights.}
  \label{tab:pillar-weights}
  \begin{tabular}{@{}llr@{}}
    \toprule
    \textbf{Pillar} & \textbf{Dimension} & \textbf{Weight} \\
    \midrule
    \pillar{Technical}            & AI crawler accessibility          & 12\% \\
    \pillar{Structure}            & Content parseability              & 18\% \\
    \pillar{Schema}               & Structured data richness          & 13\% \\
    \pillar{Authority}            & Source trustworthiness             & 12\% \\
    \pillar{Entity Recognition}   & Pre-existing AI knowledge          & 13\% \\
    \pillar{Retrieval}            & Simulated retrieval performance    & 22\% \\
    \pillar{Coverage}             & Answer completeness                & 10\% \\
    \midrule
    \textbf{Total}                &                                    & \textbf{100\%} \\
    \bottomrule
  \end{tabular}
\end{table}

\subsubsection{Pillar 1: Technical Readiness (12\%)}\label{sec:p-technical}

Measures whether AI crawlers can access the content. Constituent signals
include: HTTPS enablement (10\% of pillar), robots.txt allowing AI bots with
specificity bonus (25\%), Time to First Byte below threshold tiers (25\%),
llms.txt presence with link count bonus (20\%), and JavaScript dependency
inversely scored (20\%).

\subsubsection{Pillar 2: Semantic Structure (18\%)}\label{sec:p-structure}

Measures whether AI can parse and navigate the content. Signals include heading
hierarchy (proper H1$\to$H2$\to$H3 nesting), answer-first structure (key
information appears early), FAQ detection, link density, and content type
distribution.

\subsubsection{Pillar 3: Schema Richness (13\%)}\label{sec:p-schema}

Measures whether AI can extract structured facts. Signals include JSON-LD
presence, schema type coverage (FAQ, Article, HowTo, Product, Organization),
author attribution, freshness signals (datePublished, dateModified), and schema
validation.

\subsubsection{Pillar 4: Authority Signals (12\%)}\label{sec:p-authority}

Measures whether AI should trust this source. Signals include author
credentials, authoritative citations, original data presence, content freshness,
and E-E-A-T (Experience, Expertise, Authoritativeness, Trustworthiness)
indicators.

\subsubsection{Pillar 5: Entity Recognition (13\%)}\label{sec:p-entity}

Measures whether AI already knows who this entity is. This pillar was added to
address a systematic 23\% pessimism bias in early scoring --- well-known brands
were being scored lower than their actual citation rates because the scoring
system did not account for pre-existing AI knowledge.

Entity recognition is scored across four signal groups: Wikipedia presence and
richness (up to 30 points), Wikidata entity linkage (up to 25 points), domain
provenance including age and TLD quality (up to 20 points), and external
visibility via news mentions and search presence (up to 25 points). The raw
0--100 score maps to recognition levels: \emph{limited} ($<$25),
\emph{partial} (25--50), and \emph{full} ($>$50).

\subsubsection{Pillar 6: Retrieval Simulation (22\%)}\label{sec:p-retrieval}

The largest-weighted pillar. Uses a 4-criterion system evaluating each test
question:
\begin{equation}\label{eq:question-score}
  \text{score}(q) = 0.4 \cdot r_q + 0.4 \cdot s_q + 0.2 \cdot c_q
\end{equation}
where $r_q = \min(1.0,\; \text{rrf}_{\text{raw}} / 0.02)$ is the content
relevance, $s_q = n_{\text{found}} / n_{\text{total}}$ is the signal coverage,
and $c_q \in \{0.3, 0.6, 1.0\}$ is the answer confidence (\textsc{Low},
\textsc{Medium}, \textsc{High}).

The total simulation score combines criterion-weighted and category-weighted
components:
\begin{equation}\label{eq:sim-total}
  S_{\text{sim}} = 0.7 \cdot S_{\text{criterion}} + 0.3 \cdot S_{\text{category}}
\end{equation}

\subsubsection{Pillar 7: Answer Coverage (10\%)}\label{sec:p-coverage}

Measures what proportion of test questions the website can answer, including
overall coverage percentage, entity-fact coverage, and product/how-to coverage.

\subsection{Score Computation}\label{sec:score-computation}

The composite \findable{} is a weighted sum of pillar scores:
\begin{equation}\label{eq:findable-score}
  \boxed{\;
    F = \sum_{p=1}^{7} \frac{w_p \cdot S_p}{100}
  \;}
\end{equation}
where $S_p \in [0, 100]$ is the raw score of pillar~$p$ and $w_p$ is its
percentage weight, with the constraint $\sum_{p} w_p = 100$.

When some pillars cannot be evaluated (e.g., observation data unavailable), the
score is rescaled to preserve the full 0--100 range:
\begin{equation}\label{eq:adjusted-score}
  F_{\text{adj}} = \frac{\sum_{p \in \mathcal{E}} w_p \cdot S_p}
                        {\sum_{p \in \mathcal{E}} w_p} \cdot \frac{100}{100}
\end{equation}
where $\mathcal{E}$ denotes the set of evaluated pillars.

The resulting score maps to five findability levels, shown in
Table~\ref{tab:levels}.

\begin{table}[ht]
  \centering
  \caption{Findability levels.}
  \label{tab:levels}
  \begin{tabular}{@{}lcl@{}}
    \toprule
    \textbf{Score Range} & \textbf{Level} & \textbf{Interpretation} \\
    \midrule
    0--39   & Not Yet Findable    & AI crawlers cannot access or cite this content \\
    40--54  & Partially Findable  & Foundation in place, key signals missing \\
    55--69  & Findable            & AI can find and cite this content \\
    70--84  & Highly Findable     & Ahead of most competitors \\
    85--100 & Optimized           & Top-tier AI sourceability \\
    \bottomrule
  \end{tabular}
\end{table}

% ==============================================================================
% 4. IMPLEMENTATION
% ==============================================================================
\section{Implementation}\label{sec:implementation}

The system is implemented in Python~3.11+ with the following architecture:

\begin{itemize}
  \item \textbf{API layer}: FastAPI (async), serving audit requests and results
    via a RESTful \texttt{/v1} endpoint.
  \item \textbf{Task queue}: Redis + RQ for background audit execution.
  \item \textbf{Database}: PostgreSQL with pgvector extension for hybrid
    retrieval (BM25 + cosine similarity).
  \item \textbf{Crawling}: httpx for static pages; Playwright (headless
    Chromium) triggered when JavaScript render delta exceeds a configurable
    threshold.
  \item \textbf{Extraction}: trafilatura with BeautifulSoup fallback.
  \item \textbf{Embeddings}: bge-small-en-v1.5 via sentence-transformers,
    producing 384-dimensional normalized vectors.
  \item \textbf{Hosting}: Railway (separate API and worker services).
\end{itemize}

The audit pipeline is implemented as a linear async sequence of approximately 15
steps. Failures in optional sections (observation, calibration) do not halt the
audit --- they mark sections as \texttt{evaluated=False}. Progress tracking uses
optimistic locking (\texttt{version\_id\_col}) to prevent race conditions in
concurrent audit scenarios.

% ==============================================================================
% 5. RESULTS
% ==============================================================================
\section{Results}\label{sec:results}

\subsection{Calibration Corpus}\label{sec:corpus}

% TODO: Update with final calibration numbers after recalibration
As of February 2026, the calibration corpus contains 1,275+ samples across 35+
unique domains. Table~\ref{tab:mention-vs-citation} summarizes the critical
distinction between mentions and citations as calibration targets.

\begin{table}[ht]
  \centering
  \caption{Mention vs.\ citation as calibration targets.}
  \label{tab:mention-vs-citation}
  \begin{tabular}{@{}lccl@{}}
    \toprule
    \textbf{Target} & \textbf{Positive Rate} & \textbf{Negative Samples}
      & \textbf{Calibration Utility} \\
    \midrule
    \obs{mentioned} & 99.8\% & 2 / 865  & Useless --- trivial classifier achieves 99.8\% \\
    \obs{cited}     & 67.7\% & 279 / 865 & Meaningful --- real signal to learn from \\
    \bottomrule
  \end{tabular}
\end{table}

The explanation is straightforward: AI language models are trained on web-scale
text corpora that include virtually every recognizable brand. When asked about
any known entity, the model will mention it from parametric memory alone --- no
retrieval required. A scoring system calibrated against mentions would report
100\% findability for sites with 0\% citation rate --- a useless vanity metric.

\subsection{Optimization}\label{sec:optimization}

We use grid search optimization with the following search space:
\begin{itemize}
  \item 7 pillar weights, each in $[5, 35]$ with step~5, constrained to
    $\sum w_p = 100$ (${\sim}4{,}500$ valid combinations).
  \item Findability threshold: $\{25, 30, 35, 40, 45, 50, 55, 60\}$.
  \item Source primacy bonus weight: $\{0, 5, 10, 15, 20\}$.
\end{itemize}

We evaluate using the Matthews Correlation Coefficient (MCC):
\begin{equation}\label{eq:mcc}
  \text{MCC} = \frac{\text{TP} \cdot \text{TN} - \text{FP} \cdot \text{FN}}
    {\sqrt{(\text{TP}+\text{FP})(\text{TP}+\text{FN})(\text{TN}+\text{FP})(\text{TN}+\text{FN})}}
\end{equation}
MCC ranges from $-1$ (worst) to $+1$ (perfect) and correctly handles class
imbalance --- a trivial ``predict all positive'' classifier scores
$\text{MCC} = 0$, not accuracy~$= 68\%$.

\paragraph{Domain-stratified validation.}
Calibration samples are split by \emph{domain}, not randomly. All samples from a
given domain go entirely into training or entirely into holdout --- never split
between them. This forces the model to predict citation for \emph{unseen
domains} based on structural properties alone. Constraints: minimum 10 training
domains, minimum 3 holdout domains, temporal ordering preserved.

\subsection{Calibration Results}\label{sec:calibration-results}

% TODO: Update with final calibration numbers after recalibration
Table~\ref{tab:calibration-results} shows current calibration performance at
threshold~$= 30$.

\begin{table}[ht]
  \centering
  \caption{Calibration results (threshold $= 30$, default pillar weights).}
  \label{tab:calibration-results}
  \begin{tabular}{@{}lcc@{}}
    \toprule
    \textbf{Metric} & \textbf{Training Set} & \textbf{Holdout Set} \\
    \midrule
    Accuracy             & 80.8\% & 50.7\% \\
    Over-prediction rate & 19.2\% & 49.3\% \\
    Under-prediction rate & 0.0\% & 0.0\% \\
    MCC                  & ${\sim}0.15$ & ${\sim}0.01$ \\
    \bottomrule
  \end{tabular}
\end{table}

The optimized weights shift substantially from defaults (Table~\ref{tab:weight-shift}),
with authority and coverage increasing while technical and structural signals
decrease.

\begin{table}[ht]
  \centering
  \caption{Weight shift from default to optimized.}
  \label{tab:weight-shift}
  \begin{tabular}{@{}lcc@{}}
    \toprule
    \textbf{Pillar} & \textbf{Default} & \textbf{Optimized} \\
    \midrule
    Technical           & 12\% & 5\%  \\
    Structure           & 18\% & 5\%  \\
    Schema              & 13\% & 20\% \\
    Authority           & 12\% & 25\% \\
    Entity Recognition  & 13\% & 5\%  \\
    Retrieval           & 22\% & 15\% \\
    Coverage            & 10\% & 25\% \\
    \bottomrule
  \end{tabular}
\end{table}

\subsection{The Citation Paradox}\label{sec:citation-paradox}

Per-domain analysis revealed a counterintuitive pattern
(Table~\ref{tab:paradox}): low-citation sites have \emph{higher} pillar scores
on average than high-citation sites.

\begin{table}[ht]
  \centering
  \caption{The Citation Paradox: pillar scores by citation tier.}
  \label{tab:paradox}
  \begin{tabular}{@{}lccc@{}}
    \toprule
    \textbf{Citation Tier} & \textbf{Avg Authority} & \textbf{Avg Coverage}
      & \textbf{Avg Technical} \\
    \midrule
    HIGH ($>$60\% cited) & 54 & 64 & 78 \\
    LOW ($<$30\% cited)  & 60 & 95 & 82 \\
    \bottomrule
  \end{tabular}
\end{table}

SaaS marketing sites (Datadog, Intercom, HubSpot) tend to have excellent
technical infrastructure but produce \emph{generic} content covered by many
competitors. Documentation sites (docs.python.org, MDN) often have worse
technical scores but produce \emph{canonical} content that AI must cite because
no equivalent source exists.

\begin{figure}[ht]
  \centering
  \includegraphics[width=0.85\textwidth]{figures/figure_1_score_distribution.pdf}
  \caption{Distribution of \findable{} scores across the calibration corpus,
    showing the separation between cited and uncited domains.}
  \label{fig:score-distribution}
\end{figure}

\begin{figure}[ht]
  \centering
  \includegraphics[width=0.85\textwidth]{figures/figure_2_citation_by_site_type.pdf}
  \caption{Citation rates by site type. Documentation and reference sites
    achieve near-universal citation, while SaaS marketing and news sites exhibit
    the widest variance.}
  \label{fig:citation-by-type}
\end{figure}

\subsection{Content Type as Dominant Predictor}\label{sec:content-type}

Across the calibration corpus, content type is the single strongest predictor of
citation behavior (Table~\ref{tab:content-type-citation}).

\begin{table}[ht]
  \centering
  \caption{Citation rates by content type.}
  \label{tab:content-type-citation}
  \begin{tabular}{@{}lcc@{}}
    \toprule
    \textbf{Content Type} & \textbf{Citation Rate} & \textbf{$N$} \\
    \midrule
    Documentation    & 100\%     & 16 \\
    Reference        & 90\%+    & 15 \\
    Developer Tools  & 84\%     & 51 \\
    SaaS Marketing   & 0--68\%  & ${\sim}200$ \\
    News Media       & 0--20\%  & ${\sim}60$ \\
    \bottomrule
  \end{tabular}
\end{table}

\begin{figure}[ht]
  \centering
  \includegraphics[width=0.85\textwidth]{figures/figure_3_pillar_distributions.pdf}
  \caption{Distribution of individual pillar scores across citation tiers,
    illustrating the paradox that low-citation sites score higher on multiple
    pillars.}
  \label{fig:pillar-distributions}
\end{figure}

% ==============================================================================
% 5b. THE TWO-LAYER MODEL
% ==============================================================================
\subsection{Site Type Classification}\label{sec:site-types}

To resolve the Citation Paradox, we classify websites into 9 content types with
empirically-derived citation baselines (Table~\ref{tab:site-types}).

\begin{table}[ht]
  \centering
  \caption{Site type classification with empirical citation baselines.}
  \label{tab:site-types}
  \begin{tabular}{@{}lcll@{}}
    \toprule
    \textbf{Site Type} & \textbf{Baseline} & \textbf{Range} & \textbf{Examples} \\
    \midrule
    Documentation   & 95\%  & 85--100\% & docs.python.org, tailwindcss.com \\
    Reference       & 85\%  & 75--95\%  & Wikipedia, MDN, W3C, IETF \\
    Developer Tools & 80\%  & 65--95\%  & GitHub, GitLab, Vercel, NPM \\
    Blog            & 50\%  & 20--80\%  & Variable by content uniqueness \\
    SaaS Marketing  & 45\%  & 0--70\%   & HubSpot, Datadog, Intercom \\
    Ecommerce       & 35\%  & 15--55\%  & Amazon, Etsy, Shopify \\
    UGC Platform    & 20\%  & 5--35\%   & Reddit, Quora, Yelp \\
    News Media      & 10\%  & 0--25\%   & BBC, CNN, WSJ, Reuters \\
    Mixed           & 50\%  & 20--80\%  & No dominant content type \\
    \bottomrule
  \end{tabular}
\end{table}

Detection is deterministic (no ML required): domain pattern matching
(\texttt{docs.*}, \texttt{*.readthedocs.io} $\to$ Documentation), page type
distribution ($>$60\% documentation pages $\to$ Documentation), URL pattern
analysis, and predominant Schema.org markup type.

\subsection{Source Primacy}\label{sec:source-primacy}

Source primacy measures whether a website is \emph{the} canonical source for its
content:

\begin{table}[ht]
  \centering
  \caption{Source primacy levels with expected citation rates.}
  \label{tab:source-primacy}
  \begin{tabular}{@{}lccc@{}}
    \toprule
    \textbf{Level} & \textbf{Primacy Score} & \textbf{Expected Citation}
      & \textbf{Example} \\
    \midrule
    \textsc{Primary}       & $\geq 0.65$ & ${\sim}90\%$ & docs.python.org for Python \\
    \textsc{Authoritative}  & 0.40--0.65  & ${\sim}65\%$ & MDN for web standards \\
    \textsc{Derivative}     & $< 0.40$    & ${\sim}25\%$ & Generic ``best tools'' blog \\
    \bottomrule
  \end{tabular}
\end{table}

Five scoring signals (each producing a value in $[-1.0, +1.0]$) are combined:
site type signal, content type distribution, URL pattern analysis, domain
alignment, and self-reference ratio. The raw score is normalized:
\begin{equation}\label{eq:primacy}
  P = \text{clamp}\!\left(\frac{\sum_{i=1}^{5} s_i + 5.0}{10.0},\; 0,\; 1\right)
\end{equation}
where $s_i \in [-1.0, +1.0]$ for each of the five signals.

\subsection{The Two-Layer Model}\label{sec:two-layer}

The complete framework separates two distinct questions
(Table~\ref{tab:two-layer}):

\begin{table}[ht]
  \centering
  \caption{The two-layer model.}
  \label{tab:two-layer}
  \begin{tabular}{@{}lp{3.5cm}p{3.5cm}p{3.5cm}@{}}
    \toprule
    \textbf{Layer} & \textbf{Question} & \textbf{Measures} & \textbf{Methodology} \\
    \midrule
    \findable{} & ``Can AI find and use your content?''
      & 7 structural pillars (what the website controls)
      & Deterministic simulation \\
    \addlinespace
    Citation Context & ``Will AI actually cite your URL?''
      & Site type, source primacy, question-category alignment
      & Empirical baselines + classification \\
    \bottomrule
  \end{tabular}
\end{table}

This separation prevents conflating optimization advice. A SaaS marketing site
with a \findable{} of 80/100 is technically excellent --- but its Citation
Context may predict only 45\% citation baseline. The correct advice is not
``improve your technical scores'' (already high) but ``create
documentation-style content that AI must cite as canonical.''

% ==============================================================================
% 6. OBSERVATION SYSTEM
% ==============================================================================
\section{Observation System}\label{sec:observation}

\subsection{Reality Snapshots}\label{sec:snapshots}

The observation pathway queries live AI models with the same test questions used
in simulation, then parses responses for mentions and citations using
regex-based detection.

Citation depth is measured on a 0--5 scale (Table~\ref{tab:citation-depth}).
Depth~$\geq 3$ constitutes the \emph{citable threshold} --- the point at which
the AI response directs users to the website.

\begin{table}[ht]
  \centering
  \caption{Citation depth scale.}
  \label{tab:citation-depth}
  \begin{tabular}{@{}clp{7cm}@{}}
    \toprule
    \textbf{Depth} & \textbf{Meaning} & \textbf{Example} \\
    \midrule
    0 & Not mentioned       & No reference to brand or content \\
    1 & Brand mentioned     & ``Datadog is a monitoring platform'' \\
    2 & Listed as option    & ``Options include Datadog, New Relic, and Grafana'' \\
    3 & Cited with URL      & ``See Datadog's docs at datadog.com/docs'' \\
    4 & Recommended         & ``We recommend using Datadog for this use case'' \\
    5 & Authoritative source & ``According to Datadog's official documentation\ldots'' \\
    \bottomrule
  \end{tabular}
\end{table}

\subsection{Divergence Analysis}\label{sec:divergence}

Comparing simulated predictions against observed reality produces three
divergence metrics: prediction accuracy (percentage of questions where
simulation correctly predicted citation), optimism bias (predicted citation but
none observed), and pessimism bias (predicted no citation but citation observed).
Systematic divergence patterns reveal calibration drift as AI models update.

\begin{figure}[ht]
  \centering
  \includegraphics[width=0.85\textwidth]{figures/figure_4_confusion_matrix.pdf}
  \caption{Confusion matrix for simulation predictions vs.\ observed citations
    on the holdout set.}
  \label{fig:confusion-matrix}
\end{figure}

\begin{figure}[ht]
  \centering
  \includegraphics[width=0.85\textwidth]{figures/figure_5_citation_by_category.pdf}
  \caption{Citation rates by question category. Differentiation and identity
    questions show the highest citation rates, while contact and trust questions
    are rarely citation-generating.}
  \label{fig:citation-by-category}
\end{figure}

% ==============================================================================
% 7. FIX GENERATION
% ==============================================================================
\section{Fix Generation}\label{sec:fixes}

Actionable remediation is generated in three tiers:

\paragraph{Tier C --- Pattern matching (instant).}
Pre-computed fix templates triggered by specific pillar scores or signal
patterns. Example: Schema score $< 30$ $\to$ ``Add FAQ schema to your top 5
pages.'' Each fix includes an estimated point impact.

\paragraph{Tier B --- Synthetic patch (seconds).}
Generates synthetic content addressing a specific deficiency, then re-scores
only the affected questions to estimate impact without full re-audit cost.

\paragraph{Tier A --- Full re-score (minutes).}
Post-implementation verification via complete audit pipeline re-run.

Fixes are organized into an \textbf{Action Center} with three views: quick wins
(low effort, high impact), high priority (critical blocking issues), and path to
next milestone (minimum fix set to reach the next findability level). Each fix
is tagged with a standardized reason code (18 codes across 5 categories:
content gaps, structure issues, quality issues, trust gaps, technical problems).

% ==============================================================================
% 8. DISCUSSION
% ==============================================================================
\section{Discussion}\label{sec:discussion}

\subsection{Why 50\% Holdout Accuracy Is Honest}\label{sec:honest-accuracy}

The 30-point gap between training (80.8\%) and holdout (50.7\%) accuracy is not
a failure --- it is a meaningful finding. It demonstrates that
\textbf{website-side structural features alone are insufficient to fully predict
AI citation behavior}. Citation depends on factors outside the website's
control:

\begin{enumerate}
  \item \textbf{AI model training data}: Whether the model has seen the
    website's content during pre-training.
  \item \textbf{Retrieval index composition}: Which websites are in the AI
    engine's retrieval index at query time.
  \item \textbf{Query specificity}: Broad queries cite fewer sources than
    specific technical questions.
  \item \textbf{Competitive landscape}: Whether alternative sources exist for
    the same information.
  \item \textbf{Model behavior drift}: AI models update their citation patterns
    over time.
\end{enumerate}

The \findable{} measures what the website \emph{can control} --- structural
readiness for retrieval and citation. The Citation Context layer provides
probabilistic prediction of what will \emph{actually happen} given factors
outside the website's control.

\subsection{Limitations}\label{sec:limitations}

\begin{enumerate}
  \item \textbf{Corpus size}: 35+ domains is sufficient for initial calibration
    but insufficient for robust per-site-type weight profiles. We estimate 100+
    diverse domains are needed.
  \item \textbf{Model coverage}: Observations currently use a single LLM
    provider per audit. Multi-model observation would improve robustness.
  \item \textbf{Temporal stability}: AI models update frequently. Citation
    patterns observed today may not hold in 3 months. Continuous recalibration
    is required.
  \item \textbf{English-only}: All universal questions, chunking heuristics, and
    citation detection patterns are English-language.
  \item \textbf{Simulation fidelity}: Our retrieval simulation uses the same
    class of techniques as production RAG systems but cannot replicate the exact
    retrieval behavior of any specific AI answer engine.
\end{enumerate}

\subsection{Ethical Considerations}\label{sec:ethics}

Publishing this methodology creates a dual-use concern: it enables both
legitimate optimization (making useful content more findable) and potential
gaming (optimizing low-quality content to appear findable). We mitigate this
through: pillar design that rewards genuinely useful structural properties,
authority and entity recognition pillars that are difficult to artificially
inflate, and continuous recalibration against observed reality that detects and
corrects for gaming patterns.

% ==============================================================================
% 9. CONCLUSION
% ==============================================================================
\section{Conclusion}\label{sec:conclusion}

We have presented the \findable{}, a transparent methodology for measuring AI
sourceability. The key contributions are:

\begin{enumerate}
  \item \textbf{The mention/citation distinction}: A fundamental measurement
    correction that transforms AI visibility tracking from a vanity metric
    (99.8\% positive) to an actionable signal (68\% positive).
  \item \textbf{The two-layer model}: Separating sourceability (what you
    control) from citation likelihood (what depends on context) provides honest,
    actionable guidance rather than false precision.
  \item \textbf{The Citation Paradox resolution}: Site type classification and
    source primacy explain why technically excellent sites can receive zero
    citations --- and what to do about it.
  \item \textbf{Transparent methodology}: Every algorithm, threshold, weight,
    and formula is disclosed.
\end{enumerate}

The 50.7\% holdout accuracy represents both the current state and the honest
frontier of website-side citation prediction. Improving beyond this requires
features beyond structural analysis --- including AI model behavior modeling,
real-time retrieval index monitoring, and competitive content analysis --- which
we propose as future work.
% TODO: Update with final calibration numbers after recalibration

% ==============================================================================
% REFERENCES
% ==============================================================================
\bibliographystyle{plainnat}
\bibliography{references}

% ==============================================================================
% APPENDICES
% ==============================================================================
\appendix

\section{Default Pillar Weights}\label{app:weights}

\begin{table}[ht]
  \centering
  \begin{tabular}{@{}lr@{}}
    \toprule
    \textbf{Pillar} & \textbf{Weight} \\
    \midrule
    technical           & 12\% \\
    structure           & 18\% \\
    schema              & 13\% \\
    authority           & 12\% \\
    entity\_recognition & 13\% \\
    retrieval           & 22\% \\
    coverage            & 10\% \\
    \midrule
    \textbf{Total}      & \textbf{100\%} \\
    \bottomrule
  \end{tabular}
\end{table}

\section{Findability Threshold}\label{app:threshold}

Default threshold: 30. Lowered from initial 50 based on calibration results
showing threshold~$= 30$ maximizes training accuracy without excessive
over-prediction on the holdout set.

\section{Reason Code Taxonomy}\label{app:reason-codes}

\begin{table}[ht]
  \centering
  \small
  \caption{Complete reason code taxonomy for fix generation.}
  \label{tab:reason-codes}
  \begin{tabular}{@{}llll@{}}
    \toprule
    \textbf{Category} & \textbf{Code} & \textbf{Severity} & \textbf{Impact} \\
    \midrule
    Content   & MISSING\_DEFINITION    & Critical & 0.30 \\
    Content   & MISSING\_PRICING       & High     & 0.25 \\
    Content   & MISSING\_FEATURES      & High     & 0.20 \\
    Content   & MISSING\_CONTACT       & High     & 0.20 \\
    Content   & MISSING\_LOCATION      & Medium   & 0.15 \\
    Content   & MISSING\_SOCIAL\_PROOF & Medium   & 0.15 \\
    \addlinespace
    Structure & BURIED\_ANSWER         & Medium   & 0.15 \\
    Structure & FRAGMENTED\_INFO       & Medium   & 0.10 \\
    Structure & NO\_DEDICATED\_PAGE    & Medium   & 0.15 \\
    Structure & POOR\_HEADINGS         & Low      & 0.10 \\
    \addlinespace
    Quality   & NOT\_CITABLE           & Medium   & 0.10 \\
    Quality   & VAGUE\_LANGUAGE        & Medium   & 0.10 \\
    Quality   & OUTDATED\_INFO         & High     & 0.20 \\
    Quality   & INCONSISTENT           & Critical & 0.25 \\
    \addlinespace
    Trust     & TRUST\_GAP             & Medium   & 0.15 \\
    Trust     & NO\_AUTHORITY          & Medium   & 0.10 \\
    Trust     & UNVERIFIED\_CLAIMS     & Medium   & 0.10 \\
    \addlinespace
    Technical & RENDER\_REQUIRED       & High     & 0.20 \\
    Technical & BLOCKED\_BY\_ROBOTS    & Critical & 0.30 \\
    \bottomrule
  \end{tabular}
\end{table}

\end{document}
